\documentclass[conference]{IEEEtran}
\IEEEoverridecommandlockouts
% The preceding line is only needed to identify funding in the first footnote. If that is unneeded, please comment it out.
\usepackage{cite}
\usepackage{amsmath,amssymb,amsfonts}
\usepackage{algorithmic}
\usepackage{graphicx}
\usepackage{textcomp}
\usepackage{xcolor}
\def\BibTeX{{\rm B\kern-.05em{\sc i\kern-.025em b}\kern-.08em
    T\kern-.1667em\lower.7ex\hbox{E}\kern-.125emX}}
\begin{document}

\title{Registration and Engineering Setup Report\\
}

\author{\IEEEauthorblockN{Yulin Zhang}
\IEEEauthorblockA{\textit{7th Group} \\
\textit{Software Engineering}\\
Montreal, Canada \\
silveralex2023820@gmail.com}
\and
\IEEEauthorblockN{Yuhang Chen}
\IEEEauthorblockA{\textit{7th Group} \\
\textit{Software Engineering}\\
Montreal, Canada \\
yuhang.chen@mail.concordia.ca}
\and
\IEEEauthorblockN{Jiaxi Yang}
\IEEEauthorblockA{\textit{7th Group} \\
\textit{Software Engineering}\\
Montreal, Canada \\
yjxyang2@outlook.com}
\and
\IEEEauthorblockN{Boyang Wang}
\IEEEauthorblockA{\textit{7th Group} \\
\textit{Software Engineering}\\
Montreal, Canada \\
wangboyang0626@outlook.com}
}

\maketitle

\begin{abstract}
This document is a report of the members and development tools used in 7th group.
\end{abstract}

\begin{IEEEkeywords}
members, development tools
\end{IEEEkeywords}



\section{Team Members}

In this session we introduce our group members in no particular order.

\subsection{Yulin Zhang}

\textbf{Name}: Yulin Zhang

\textbf{SID}: 40264421

\textbf{Program}:Meng. Electrical and Computer Engineering

\textbf{Email}: silveralex2023820@gmail.com

\textbf{SE Background}: I specialize in computer graphics algorithms, game development, and software architecture, and I am good at C++ and Python. I used to work as a game engine programmer in Netease for 2.5 years, focusing on graphics algorithm development, such as real-time global illumination algorithms. I also have research experience in ultrasonic image diagnosis by deep learning and have published two SCI papers. I am eager to dedicate myself to the video game industry in Montreal.



\subsection{Yuhang Chen}


\textbf{Name}: Yuhang Chen

\textbf{SID}: 40253925

\textbf{Program}:Meng. Electrical and Computer Engineering

\textbf{Email}: yuhang.chen@mail.concordia.ca

\textbf{SE Background}: Master Python, familiar with Pycharm, VSCode, MATLAB, MySQL, and PostgreSQL. Familiar with the whole set of processes to build a GIS (Geographic information system). Ability to build website front-end interfaces using Vue. Worked for Tencent and Huawei as a software development engineer-testing intern. Interesting in computer vision and software development.


\subsection{Jiaxi Yang}


\textbf{Name}:  Jiaxi Yang

\textbf{SID}:  40261989

\textbf{Program}:Meng. Electrical and Computer Engineering

\textbf{Email}: yjxyang2@outlook.com

\textbf{SE Background}: I specialize in front-end web systems and back-end systems, and I have mastered many languages, such as C, C++, Java, and Python. I am familiar with libs such as Spring, HTML, and CSS. Meanwhile, I have a lot of experience with PyCharm, Visual Studio Code, Visual Studio, IntelliJ IDEA, MySQL Workbench, and Microsoft Visio. I used to be a backend developer at Luo Yang Junda Real Estate Co., Ltd. for three months. I also have research experience in artificial intelligence, which resulted in 1 EI Conference Paper, 2 National Invention patents, and one more journal manuscript. Finnaly, I have a great passion for AI, Deep Learning, Computer Vision, and Web Development.


\subsection{Boyang Wang}

\textbf{Name}: Boyang Wang

\textbf{SID}: 40274468

\textbf{Program}:Meng. Electrical and Computer Engineering

\textbf{Email}: wangboyang0626@outlook.com

\textbf{SE Background}: I am good at C and Python. I used to work on a project about student performance management system. I also have some experience with embedded system. I am interested on software development industry.



\section{Development Tools}

In this session we introduce the tools used in the project development.


\subsection{Version Control}

We use \textbf{Git} as our version control tool. The graphics interface of Git makes it easy to control the project version. We have built a repository on GitHub for our group, which ensures the steady advancement of our project.

In summary, our version control tool is \textbf{Git}.
 
\subsection{Programming Languages}

The project can be divided into two parts: front-end and back-end. As for the front-end, we use \textbf{HTML} to create the web page and \textbf{Javascript} to control the behavior of web pages. As for the back-end, we use \textbf{Python} to process the analysis logic. Python is a powerful language, which has a lot of development frameworks for us to boost our project progress, such as Flask. 

In summary, we will use \textbf{HTML}, \textbf{Javascript}, and \textbf{Python} in this project.

\subsection{IDE \& Development Tools}

Based on the languages we used in this project, we have chosen \textbf{Pycharm} as our main IDE to develop a Python back-end program.
 As for the database, we chose \textbf{MongoDB} for the data storage. Meanwhile, We decided to build up the project based on \textbf{Flask} framework.
 
 
 In summary, \textbf{Pycharm} is our main IDE, and we will develop the project based on \textbf{MongoDB} and \textbf{Flask}.
 

\subsection{Group Communication Software}

We communicate with each other by \textbf{Wechat}. We are familiar with the usage of this app, and we have built up a development group for our communication.

In summary, we use \textbf{Wechat} as our group communication software.

\subsection{Tracking Tools}

Because we are a small group and have the requirement of swift development. We chose \textbf{Monday} (https://monday.com/) as our tracking tool to advance our project. Monday is suitable for small development groups, we can create an event easily.

In summary, we use \textbf{Monday} as our tracking tool.


\end{document}
